% — TestXeTeX_SetAngsana.tex——-
% — Chakkree Tiyawongsuwan —-
%
% อ้างอิง http://markmail.org/message/ulp3lmxfg5biv7b5
%
% Modified by Kittipong Piyawanno

\documentclass[a4paper, oneside]{article}
\usepackage{fontspec}
\usepackage{xunicode}
\usepackage{xltxtra}
\usepackage{graphicx}
\usepackage{wallpaper}
\usepackage{ulem}

\XeTeXlinebreaklocale “th_TH”
\defaultfontfeatures{Scale=1.6}
\renewcommand{\baselinestretch}{1.8}
\setmainfont{TH SarabunPSK:script=thai}
\setmainfont[BoldFont={TH SarabunPSK Bold:script=thai},
	ItalicFont={TH SarabunPSK Italic:script=thai},
	BoldItalicFont={TH SarabunPSK Bold Italic:script=thai}
]{TH SarabunPSK:script=thai}
\setsansfont{TH SarabunPSK:script=thai}
\setmonofont[Scale=0.8]{Tahoma:script=thai}

\setlength{\textwidth}{16cm}
\setlength{\textheight}{23.5cm}
\setlength{\oddsidemargin}{0.4cm}
\setlength{\evensidemargin}{0mm}
\setlength{\topmargin}{-1.5cm}
\setlength{\headheight}{0.6cm}
\setlength{\headsep}{1cm}
\setlength{\parskip}{6pt}
\setlength{\parindent}{25mm}
\thispagestyle{empty}

\begin{document}
	\ThisCenterWallPaper{1.0}{GeneralBG.pdf}
	\ 
	\vskip 0.5cm
	\
	\newline
	\makebox[11.0cm][l]{ที่ นร ๐๑๐๖/ว ๒๐๑๙}
	\makebox[5.0cm][l]{สำนักนายกรัฐมนตรี}
	\newline
	\makebox[11.0cm][l]{}
	\makebox[5.0cm][l]{ทำเนียบรัฐบาล กทม. ๑๐๓๐๐}
	
	\makebox[6.0cm][l]{}
	\makebox[5.0cm][l]{๓๐ พฤศจิกายน ๒๕๕๓}
	
	\hskip -2.7cm
	\begin{tabular}{p{0.6cm} p{15.4cm}}
		เรื่อง &
		คำอธิบายการพิมพ์หนังสือราชการภาษาไทยด้วยโปรแกรมการพิมพ์ในเครื่องคอมพิวเตอร์และตัวอย่างการพิมพ์ \\
		เรียน &
		ปลัดกระทรวงกลาโหม \\
	\end{tabular}
	
	\hskip -2.7cm
	\begin{tabular}{p{1.0cm} p{15.0cm}}
		อ้างถึง &
		หนังสือสำนักเลขาธิการนายกรัฐมนตรี ด่วนที่สุด ที่ นร ๐๕๐๖/ว ๑๖๔ ลงวันที่ ๑๓ กันยายน ๒๕๕๓ \\
	\end{tabular}
	
	\hskip -2.7cm
	\begin{tabular}{p{2.2cm} p{13.8cm}}
		สิ่งที่ส่งมาด้วย&สำเนาคำอธิบายการพิมพ์หนังสือราชการภาษาไทยด้วยโปรแกรมการพิมพ์ในเครื่องคอมพิวเตอร์ และตัวอย่างการพิมพ์หนังสือจำนวน ๓ แบบ
	\end{tabular}
	
	ตามหนังสือที่อ้างถึง สำนักเลขาธิการคณะรัฐมนตรีได้แจ้งมติคณะรัฐมนตรีเมื่อวันที่ ๗ กันยายน ๒๕๕๓
	ซึ่งเห็นชอบให้หน่วยงานภาครัฐทุกหน่วยดำเนินการติดตั้งรูปแบบตัวพิมพ์ (ฟอนต์) สารบรรณและรูปแบบตัวพิมพ์
	(ฟอนต์) อื่น ๆ ทั้งหมดจำนวน ๑๓ รูปแบบตัวพิมพ์ของสำนักงานส่งเสริมอุตสาหกรรมซอพท์แวร์แห่งชาติ
	(องค์การมหาชน) และกรมทรัพย์สินทางปัญญา เพิ่มเข้าไปในรูปแบบปฏิบัติการ Thai OS และใช้รูปแบบ
	ตัวพิมพ์ดังกล่าว แทนรูปแบบตัวพิมพ์เดิม เพื่อให้เอกสารของส่วนราชการเป็นไปอย่างมีมาตรฐาน ไม่มีปัญหา
	ละเมิดลิขสิทธิ์และไม่ขึ้นกับระบบปฏิบัติการใดระบบปฏิบัติการหนึ่ง ตามที่กระทรวงเทคโนโลยีสารสนเทศและการสื่อ- สารได้เสนอ ความละเอียดแจ้งแล้วนั้น
	
	อาศัยอำนาจตามระเบียบสำนักนายกรัฐมนตรี ว่าด้วยงานสารบรรณ พ.ศ. ๒๕๒๖ ข้อ ๘ ปลัดสำนักนายกรัฐมนตรี
	จึงได้จัดทำคำอธิบายการพิมพ์หนังสือราชการภาษาไทยด้วยโปรแกรมการพิมพ์ในเครื่องคอม- พิวเตอร์
	เป็นคำอธิบายเพิ่มเติมต่อจากคำอธิบายการพิมพ์หนังสือราชการภาษาไทยด้วยเครื่องพิมพ์ดีดในคำอธิบาย ๔
	ท้ายระเบียบสำนักนายกรัฐมนตรี ว่าด้วยงานสารบรรณ พ.ศ. ๒๕๒๖ และที่แก้ไขเพิ่มเติม พร้อมด้วยตัวอย่าง
	การพิมพ์หนังสือด้งกล่าว เพื่อให้ส่วนราชการได้ถือเป็นแนวทางปฏิบัติในการทำหนังสือราชการและการพิมพ์หนังสือราชการที่มีรูปแบบมาตรฐานเดียวกันตามนัยมติคณะรัฐมนตรีดังกล่าว
	ดังปรากฏตามสิ่งที่ส่งมาด้วย
	สำหรับการจัดทำหนังสือราชการตามแบบท้ายระเบียบฯ จำนวน ๑๑ แบบ เห็นควรใช้รูปแบบตัวพิมพ์ไทยสารบรรณ
	(Th Sarabun PSK)ขนาด ๑๖ พอยท์ อันจะเป็นประโยชน์ต่การพัฒนาระบบจัดเก็บข้อมูลข่าวสารหรือ
	หนังสือราชการในระบบอิเล็กทรอนิกส์ของส่วนราชการต่อไป ทั้งนี้ สามารถดาวน์โหลดแผ่นแบบ(Template)
	มาตรฐานการพิมพ์หนังสือราชการภายนอก หนังสือภายในและหนังสือประทับตรา ได้ที่ www.opm.go.th
	
	จึงเรียนมาเพื่อโปรดทราบ และกรุณาแจ้งให้ส่วนราชการในสังกัดทราบ และถือฏิบัติต่อไป
	
	\vskip 12pt

	\makebox[5.5cm][l]{}
	\makebox[4.0cm][c]{ขอแสดงความนับถือ}
	
	\
	\newline
	\makebox[8.0cm][l]{}
	\makebox[4.0cm][c]{นายจตุรงค์ ปัญญาดิลก}
	\newline
	\makebox[8.0cm][l]{}
	\makebox[4.0cm][c]{ปลัดสำนักนายกรัฐมนตรี}
	\newline
	สำนักงานปลัดสำนักนายกรัฐมนตรี
	\newline
	สำนักกฎหมายและระเบียบกลาง
	\newline
	โทร. ๐ ๒๒๘๒ ๒๖๙๔ โทรสาร ๐ ๒๒๘๒ ๗๘๙๖
	
\end{document}
