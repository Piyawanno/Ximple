% — TestXeTeX_SetAngsana.tex——-
% — Chakkree Tiyawongsuwan —-
%
% อ้างอิง http://markmail.org/message/ulp3lmxfg5biv7b5
%
% Modified by Kittipong Piyawanno

\documentclass[a4paper, oneside]{article}
\usepackage{fontspec}
\usepackage{xunicode}
\usepackage{xltxtra}
\usepackage{graphicx}
\usepackage{wallpaper}
\usepackage{ulem}

\XeTeXlinebreaklocale “th_TH”
\defaultfontfeatures{Scale=1.6}
\renewcommand{\baselinestretch}{1.8}
\setmainfont{TH SarabunPSK:script=thai}
\setmainfont[BoldFont={TH SarabunPSK Bold:script=thai},
	ItalicFont={TH SarabunPSK Italic:script=thai},
	BoldItalicFont={TH SarabunPSK Bold Italic:script=thai}
]{TH SarabunPSK:script=thai}
\setsansfont{TH SarabunPSK:script=thai}
\setmonofont[Scale=0.8]{Tahoma:script=thai}

\setlength{\textwidth}{16cm}
\setlength{\textheight}{23.5cm}
\setlength{\oddsidemargin}{0.4cm}
\setlength{\evensidemargin}{0mm}
\setlength{\topmargin}{-9mm}
\setlength{\headheight}{0.6cm}
\setlength{\headsep}{1cm}
\setlength{\parskip}{6pt}
\setlength{\parindent}{25mm}
\thispagestyle{empty}

\begin{document}
	\ThisCenterWallPaper{1.0}{InternalLetterBG.pdf}
	\begin{center}
		\begin{huge}
			\textbf{บันทึกข้อความ}
		\end{huge}	
	\end{center}
	\begin{large}
		\textbf{ส่วนราชการ}
	\end{large}
	\underline{\makebox[13.5cm][l]{\ \ ฝศษ.รร.นร. (โทร.๕๓๘๔๓)}}
	\newline
	\begin{large}
		\textbf{ที่}
	\end{large}
	\underline{\makebox[7.2cm][l]{\ \ กห ๐๕๓๕.๓/}}
	\begin{large}
		\textbf{วันที่}
	\end{large}
	\underline{\makebox[7.0cm][l]{\ \ ๑๔ ก.พ.๕๔}}
	\newline
	\begin{large}
		\textbf{เรื่อง}
	\end{large}
	\underline{\makebox[15cm][l]{\ \ ขออนุมัติจัดการสอนเพิ่มเติมนอกเวลาราชการให้แก่ นนร}}

	\hskip -2.5cm
	เสนอ รร.นร.
	
	๑.  ฝศษ.รร.นร. ขออนุมัติจัดการสอนเพิ่มเติมนอกเวลาราชการให้แก่ นนร.ชั้น ๔ วฟ.๑,  ๒  ที่มีความรู้พื้นฐานวิชาการแพร่กระจายคลื่นวิทยุประจำภาคปลาย ปีการศึกษา ๒๕๕๓ ในวันอาทิตย์ที่ ๒ ม.ค.๕๔ เวลา ๑๖๐๐ – ๒๑๐๐ โดยมีค่าใช้จ่ายเป็นค่าสอน จำนวน ๒,๔๐๐.- บาท (สองพันสี่ร้อยบาทถ้วน)
	
	๒.  ฝศษ.รร.นร.ขอเสนอเพื่อกรุณาทราบดังนี้
	
	\vskip -6pt
	
	\ \ \ \ ๒.๑  ฝศษ.รร.นร.ได้ตรวจสอบตารางปฏิบัติของ นนร. ในช่องภาคปลาย ปีการศึกษา ๒๕๕๓ แล้ว เห็นว่า นนร. มีกิจกรรมที่ต้องปฏิบัติเป็นจำนวนมาก รวมถึงการเข้าร่วมการแข่งขันสัปดาห์กีฬานาวี,  สวนสนามรักษาพระองค์ ทำให้ นนร.เรียนไม่ครบชั่วโมงศึกษา ประกอบกับ นนร.ชั้น ๔  ปัจจุบันมีความรู้พื้นฐานในวิชาการแพร่กระจายคลื่นวิทยุ ค่อนข้างอ่อน และเป็น นนร.ที่สอบตกซ้ำชั้น ปี ๕๒ จำนวนหนึ่ง 
ซึ่งหากมิได้มีการสอนทบทวนให้เพิ่มเติมจากเวลาปกติ ก็อาจส่งผลให้เรียนไม่ทันเพื่อนที่มีผลการเรียนดีและ
มีโอกาสสอบตกซ้ำชั้นอีกครั้ง
	
	\vskip -6pt
	
	\ \ \ \ ๒.๒  ในการจัดสอนเพิ่มเติมให้แก่ นนร. ที่มีความรู้พื้นฐานวิชาการแพร่กระจายคลื่นวิทยุ ซึ่ง กววศ.ฝศษ.รร.นร. จะเป็นผู้ดำเนินการสอนในวันอาทิตย์ที่ ๒ ม.ค.๕๔  เวลา ๑๖๐๐ – ๑๙๐๐ โดยมี 
น.ท.ศักดา  นฤนรนาถ อาจารย์ ฝศษ.รร.นร. อาจารย์ประจำวิชาเป็นผู้สอน และในการนี้ ฝศษ.รร.นร. 
ได้ประสานกับ กรม นนร.รอ.รร.นร. แล้ว สามารถให้ นนร.ชั้น ๔ วฟ.๑,  ๒ เข้าเรียนทบทวนเพิ่มเติมตามวันเวลาดังกล่าว โดยมีค่าใช้จ่ายเป็นค่าสอน จำนวน ๒,๐๐๐.- บาท (สองพันบาทถ้วน)
	
	๓.  ฝศษ.รร.นร. พิจารณาแล้วเห็นควรอนุมัติให้จัดการสอนเพิ่มเติมนอกเวลาราชการ และอนุมัติเงินค่าสอน จำนวน ๒,๐๐๐.- บาท (สองพันบาทถ้วน) ตามข้อ ๒.๒
	
	จึงเสนอมาเพื่อโปรดพิจารณาอนุมัติ ตามข้อ ๓.
	\newline
	\newline
	\makebox[8.0cm][l]{}
	\makebox[6.0cm][l]{พล.ร.ต.}
	\newline
	\makebox[8.0cm][l]{}
	\makebox[6.0cm][c]{หน.ฝศษ.รร.นร.}
\end{document}
